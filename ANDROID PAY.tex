\documentclass[12]{article}
\usepackage[margin=1in]{geometry}

\begin{document}
\pagestyle{empty}
\title{\textbf{ ANDROID PAY}}
\date{}
\maketitle

\section{Introduction}
Android Pay is a digital wallet platform and online payment system developed by Google to power in-app and tap-to-pay purchases on mobile devices, enabling users to make payments with Android phones, tablets or watches.\\
\section{Main Body}
Basically, Android Pay is the same tap-to-pay feature of Google Wallet, except way less of a pain to use. With Google Wallet, you had to launch an app, and then type in a pin so Google could unlock your credit cards. The whole idea of “Google Wallet” was also a little confusing, since the app doubled as a peer-to-peer payment system which could funnel money to a real, physical Google Wallet card too.\\
With Android Pay, you won’t need the app. You won’t need to enter a pin. It’s built right into the operating system. If you’ve unlocked your phone, you just place it up against the credit card terminal, and boom, you’re done. It’ll even automatically prompt you to use a loyalty card or gift card if you have one.
\section{Conclusion}
Where does that leave Google Wallet? A Googler tells us that Wallet will continue as a peer-to-peer payment system, and the physical Google Wallet card will stick around too. Android Pay is for tap to pay, and Wallet is for transferring money between friends and over the internet.

\begin{thebibliography}{}
\bibitem{ANDROIDPAY1}
P. Sarah. “TechCrunch.” Internet: https://techcrunch.com/2017/05/17/google-will-now-let-users-pay-with-any-card-they-have-on-file-not-just-those-saved-in-android-pay/, Dec. 24, 2017 [Mar.08, 2018].

\bibitem{ANDROIDPAY2}
B. Pali. “Android Official Blog.” Internet: https://blog.google/products/android/android-pay-now-in-uk-new-countries-on/, May.18, 2016 [Mar.08, 2018].

\bibitem{ANDROIDPAY3} 
W. Chris. “The Verge.” Internet: www.theverge.com/2015/3/5/8152801/softcard-shutting-down-march-31, Mar. 5, 2015 [Mar.08, 2018].
\end{thebibliography}


\end{document}